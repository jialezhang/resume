% !TEX encoding = UTF-8 Unicode
%%%%%%%%%%%%%%%%%%%%%%%%%%%%%%%%%%%%%%%%%
% Wilson Resume/CV
% XeLaTeX Template
% Version 1.0 (22/1/2015)
%
% This template has been downloaded from:
% http://www.LaTeXTemplates.com
%
% Original author:
% Howard Wilson (https://github.com/watsonbox/cv_template_2004) with
% extensive modifications by Vel (vel@latextemplates.com)
%
% License:
% CC BY-NC-SA 3.0 (http://creativecommons.org/licenses/by-nc-sa/3.0/)
%
%%%%%%%%%%%%%%%%%%%%%%%%%%%%%%%%%%%%%%%%%

%----------------------------------------------------------------------------------------
%	PACKAGES AND OTHER DOCUMENT CONFIGURATIONS
%----------------------------------------------------------------------------------------

\documentclass[10pt]{article} % Default font size

\usepackage[UTF8, heading = false, scheme = plain]{ctex}

%%%%%%%%%%%%%%%%%%%%%%%%%%%%%%%%%%%%%%%%%
% Wilson Resume/CV
% Structure Specification File
% Version 1.0 (22/1/2015)
%
% This file has been downloaded from:
% http://www.LaTeXTemplates.com
%
% License:
% CC BY-NC-SA 3.0 (http://creativecommons.org/licenses/by-nc-sa/3.0/)
%
%%%%%%%%%%%%%%%%%%%%%%%%%%%%%%%%%%%%%%%%%

%----------------------------------------------------------------------------------------
%	PACKAGES AND OTHER DOCUMENT CONFIGURATIONS
%----------------------------------------------------------------------------------------

\usepackage{CJKutf8}
\usepackage[a4paper, hmargin=25mm, vmargin=30mm, top=20mm]{geometry} % Use A4 paper and set margins

\usepackage{fancyhdr} % Customize the header and footer

\usepackage{lastpage} % Required for calculating the number of pages in the document

\usepackage{hyperref} % Colors for links, text and headings

\setcounter{secnumdepth}{0} % Suppress section numbering

%\usepackage[proportional,scaled=1.064]{erewhon} % Use the Erewhon font
%\usepackage[erewhon,vvarbb,bigdelims]{newtxmath} % Use the Erewhon font
\usepackage[utf8]{inputenc} % Required for inputting international characters
\usepackage[T1]{fontenc} % Output font encoding for international characters

\usepackage{fontspec} % Required for specification of custom fonts
\setmainfont[Path = ./fonts/,
Extension = .otf,
BoldFont = Erewhon-Bold,
ItalicFont = Erewhon-Italic,
BoldItalicFont = Erewhon-BoldItalic,
SmallCapsFeatures = {Letters = SmallCaps}
]{Erewhon-Regular}

\usepackage{color} % Required for custom colors
\definecolor{slateblue}{rgb}{0.17,0.22,0.34}

\usepackage{sectsty} % Allows customization of titles
\sectionfont{\color{slateblue}} % Color section titles

\fancypagestyle{plain}{\fancyhf{}\cfoot{\thepage\ of \pageref{LastPage}}} % Define a custom page style
\pagestyle{plain} % Use the custom page style through the document
\renewcommand{\headrulewidth}{0pt} % Disable the default header rule
\renewcommand{\footrulewidth}{0pt} % Disable the default footer rule

\setlength\parindent{0pt} % Stop paragraph indentation

% Non-indenting itemize
\newenvironment{itemize-noindent}
{\setlength{\leftmargini}{0em}\begin{itemize}}
{\end{itemize}}

% Text width for tabbing environments
\newlength{\smallertextwidth}
\setlength{\smallertextwidth}{\textwidth}
\addtolength{\smallertextwidth}{-2cm}

\newcommand{\sqbullet}{~\vrule height 1ex width .8ex depth -.2ex} % Custom square bullet point definition

%----------------------------------------------------------------------------------------
%	MAIN HEADER COMMAND
%----------------------------------------------------------------------------------------

\renewcommand{\title}[1]{
{\huge{\color{slateblue}\textbf{#1}}}\\ % Header section name and color
\rule{\textwidth}{0.5mm}\\ % Rule under the header
}

%----------------------------------------------------------------------------------------
%	Skill COMMAND
%----------------------------------------------------------------------------------------

\newcommand{\skill}[6]{
\begin{tabbing}
\hspace{2cm} \= \kill
\textbf{#1} \> \href{#4}{#3} \\
\textbf{#2} \>\+ \textit{#5} \\
\begin{minipage}{\smallertextwidth}
\vspace{2mm}
#6
\end{minipage}
\end{tabbing}
\vspace{2mm}
}

%----------------------------------------------------------------------------------------
%	PROJECT GROUP COMMAND
%----------------------------------------------------------------------------------------

\newcommand{\expectgroup}[2]{
\begin{tabbing}
\hspace{5mm} \= \kill
\sqbullet \>\+ \textit{#1} \\
\begin{minipage}{\smallertextwidth}
\vspace{2mm}
#2
\end{minipage}
\end{tabbing}
}

%----------------------------------------------------------------------------------------
%	SKILL GROUP COMMAND
%----------------------------------------------------------------------------------------

\newcommand{\skillgroup}[2]{
\begin{tabbing}
\hspace{5mm} \= \kill
\sqbullet \>\+ \textbf{#1} \\
\begin{minipage}{\smallertextwidth}
\vspace{2mm}
#2
\end{minipage}
\end{tabbing}
}

%----------------------------------------------------------------------------------------
%	INTERESTS GROUP COMMAND
%-----------------------------------------------------------------------------------------

\newcommand{\interestsgroup}[1]{
\begin{tabbing}
\hspace{5mm} \= \kill
#1
\end{tabbing}
\vspace{-10mm}
}
\newcommand{\interest}[1]{\sqbullet \> \textbf{#1}\\[3pt]} % Define a custom command for individual interests

%----------------------------------------------------------------------------------------
%	TABBED BLOCK COMMAND
%----------------------------------------------------------------------------------------

\newcommand{\tabbedblock}[1]{
\begin{tabbing}
\hspace{2cm} \= \hspace{4cm} \= \kill
#1
\end{tabbing}
} % Include the file specifying document layout

%----------------------------------------------------------------------------------------

\begin{document}

%----------------------------------------------------------------------------------------
%	NAME AND CONTACT INFORMATION
%----------------------------------------------------------------------------------------

\title{张佳乐} % Print the main header

%------------------------------------------------

\parbox{0.5\textwidth}{ % First block
\begin{tabbing} % Enables tabbing
\hspace{3cm} \= \hspace{4cm} \= \kill % Spacing within the block
{\bf 邮箱} \> gkjiale@gmail.com\\ % Address line 1
% {\bf 电话} \> 13071217000\\ % Date of birth 
{\bf 所在地点} \> 湖北武汉 \\ % Nationality
{\bf 电话} \> 13071217000\\ % Date of birth 
\end{tabbing}}
\hfill % Horizontal space between the two blocks
\parbox{0.5\textwidth}{ % Second block
\begin{tabbing} % Enables tabbing
\hspace{3cm} \= \hspace{4cm} \= \kill % Spacing within the block
% {\bf Home Phone} \> +0 (000) 111 1111 \\ % Home phone
% {\bf Mobile Phone} \> +0 (000) 111 1112 \\ % Mobile phone
% {\bf Email} \> \href{mailto:john@smith.com}{john@smith.com} \\ % Email address
\end{tabbing}}

\section{ 工作期望}

\expectgroup{工作地点}
{
\textit{武汉}
}

\expectgroup{职位}
{
\textit{后台开发工程师}
}

\expectgroup{公司阶段}
{
\textit{A轮}
\textit{E轮}
}
%----------------------------------------------------------------------------------------
%	EDUCATION SECTION
%----------------------------------------------------------------------------------------

\section{教育信息}

\tabbedblock{
\bf{2004-2007} \> 华中科技大学 - 给排水科学与工程 - 学士\\[5pt]
%------------------------------------------------
}
%----------------------------------------------------------------------------------------
%	EMPLOYMENT HISTORY SECTION
%----------------------------------------------------------------------------------------

\section{ 技能信息}

\skillgroup{语言方向}
{
\textit{Ruby} - MRI 1.8.7, 1.9.2\\
\textit{ASP.NET, C\#, VB.NET}\\
\textit{PHP}\\
\textit{Java/Scala}
}

%------------------------------------------------

\skillgroup{技能方向}
{
\textit{HTML5, CSS3/SASS, JavaScript/CoffeeScript/jQuery}\\
\textit{Ruby on Rails v3.1}\\
\textit{Test:Unit, RSpec, Cucumber, Selenium} - automated testing frameworks\\
\textit{Apache/Nginx Web Servers}\\
}


%----------------------------------------------------------------------------------------
%	INTERESTS SECTION
%----------------------------------------------------------------------------------------

\section{工作经历}

\skillgroup{web开发工程师}
{
\textit{2010-2014} - 阿里巴巴\\
\textit{简介} -  xeCJK 之解决了中文支持问题,以及一些关于标点的处理,并没有提供和中文版式相关的解决方案。ctex 宏包和文档类封装了 xeCJK,同时提供了中文版式的相关支持。新版的 ctex 宏包和文档类能够自动检测用户使用的操作系统,自动选择合适的字体配置,十分方便。\\
}

\section{项目经验}

\skillgroup{ goagent}
{
\textit{2014 - 2100} \\
\textit{简介} -  xeCJK 之解决了中文支持问题,以及一些关于标点的处理,并没有提供和中文版式相关的解决方案。ctex 宏包和文档类封装了 xeCJK,同时提供了中文版式的相关支持。新版的 ctex 宏包和文档类能够自动检测用户使用的操作系统,自动选择合适的字体配置,十分方便。\\
}

%----------------------------------------------------------------------------------------
%	REFEREE SECTION
%----------------------------------------------------------------------------------------
\section{其他信息}

说说我身边的一个骗局。最厉害在于,骗子还不会被抓。
首先这伙人会在酒店租个大厅,邀请很多老年人来听养生讲座,时间也定在只有老年人起的了床的4点开始。并许诺来的人每人送一小篮鸡蛋。
老人还是喜欢占便宜的,所以第一天就有很多人来,而且真的收到了鸡蛋。
讲座内容无非是主持人自幼家贫,无父无母,由一好心人收养,现在做生意发家致富了,就来回馈大家,把自己厂生产的保健品及保健药酒便宜卖给大家,有各种保健……功能。
第一天卖的一般都是一百元左右的产品,当然第一天最多只有5%的老人买了他说的产品。
结果第二天,鸡蛋照样继续送,每个前一天买了产品的老人都收到一个红包,里面有自己前一天买药酒及保健品的钱。
这时候很多老年人心动了,差不多30%老人买了他的产品。这一天主推两三百元的产品。
第三天,鸡蛋照送,买了药酒的钱照样通过红包方式还给老人。这时至少七八成的老人都拿了压箱底的钱来买产品了,不管有没有用,反正免费的不要白不要。
通常这时候,主办方还会喊警+察或者工*商来一两回,查看一下情况,说你这产品没有问题,活动符合要求,有人报假警之类的话,然后老人们就更加相信了“警*察工\#商都说没事,肯定没事”其实警*察和工@商估计都是假冒的。
一来二去再过几天,主推上千的产品了,这时老人们已经买疯了,甚至有人借钱来买了。
结果有一天,感觉能赚差不多了,第二天就人去楼空了。
利益相关:我外婆被骗1万5,奶奶被骗3000。
%----------------------------------------------------------------------------------------

\end{document}